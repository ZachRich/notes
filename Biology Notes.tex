\documentclass[10pt, oneside]{article} 
\usepackage{amsmath, amsthm, amssymb, calrsfs, wasysym, verbatim, bbm, color, graphics, geometry, glossaries}

\geometry{tmargin=.75in, bmargin=.75in, lmargin=.75in, rmargin = .75in}  

\newcommand{\R}{\mathbb{R}}
\newcommand{\C}{\mathbb{C}}
\newcommand{\Z}{\mathbb{Z}}
\newcommand{\N}{\mathbb{N}}
\newcommand{\Q}{\mathbb{Q}}
\newcommand{\Cdot}{\boldsymbol{\cdot}}

\newtheorem{thm}{Theorem}
\newtheorem{defn}{Definition}
\newtheorem{conv}{Convention}
\newtheorem{rem}{Remark}
\newtheorem{lem}{Lemma}
\newtheorem{cor}{Corollary}


\title{Running Lecture Outline: [BIO - 131]}
\author{[Zachary Rich]}
\date{Academic Year 2019-2020}

\begin{document}

\maketitle
\tableofcontents

\vspace{.25in}

\section{January, 2020}

\subsection{Tue, Jan 28th: Intro to Organisms II: Defining Organisms}

\begin{itemize}

\item Living Organisms have 3 requirements
\begin{itemize}
	\item Made up of one or more cells
	\begin{itemize}
		\item All organisms are made up of cells that are surrounded by a cell/plasma membrane
		\item Unicellular (aka single-celled) organisms are composed of only one cell
			\begin{itemize}
				\item Easy and quick to reproduce and have Low energy requirements
			\end{itemize}
		\item Multicellular organisms are composed of many cells connected together
			\begin{itemize}
				\item Larger body size, Specialized tissues and body parts, More control over body processes and position in environment.
			\end{itemize}
	\end{itemize}
	\item Made up of Genetic Material
		\begin{itemize}
			\item DNA!
			\begin{itemize}
				\item DNA contains genetic information
				\item Gene are the segments of DNA that contain instructions to make a specific protein
				\item Proteins influence the functions and physical characteristics of organisms.
			\end{itemize}
			\item Prokaryotes Vs. Eukaryotes
			\begin{itemize}
				\item Protists, Plants, Animals, and Fungi are Eukaryotes: DNA is contained in a nucleus inside the cell membrane
				\item Bacteria and Archaea are Prokaryotes: DNA is free-floating inside the cell membrane
			\end{itemize}
		\end{itemize}
			
	\item The ability to gather materials and harness energy from the environment
	\begin{itemize}
		\item Photoautotrophs
		\begin{itemize}
			\item Use energy from the sun to make own food
			\item Bacteria, Protists, Plantae
		\end{itemize}
		\item Heterotrophs
		\begin{itemize}
			\item Consume other autotrophs or heterotrophs to use as food
			\item Bacteria, Archaea, Fungi, Protists, Animals
		\end{itemize}
		\item Chemoautotrophs
		\begin{itemize}
			\item Use energy from chemicals in the environment to make own food
			\item Bacteria, Archaea
		\end{itemize}
\end{itemize}
\end{itemize}
\end{itemize}

\begin{itemize}
	\item 6 Major Groups of Organisms
	
	\begin{itemize}
	\item \textbf{\emph{Animals}}
	
	\begin{itemize}
		\item MultiCellular only
		\item Eukaryotes only
		\item Hetero-
	\end{itemize}
	\item \textbf{\emph{Plants}}
	
	\begin{itemize}
		\item MultiCellular only
		\item Eukaryotes only
		\item Photoauto-
	\end{itemize}
	\item \textbf{\emph{Fungi}}
	
	\begin{itemize}
		\item Multicellular and Unicellular
		\item Eukaryotes only
		\item Hetero-
	\end{itemize}
	\item \textbf{\emph{Bacteria}}
	
	\begin{itemize}
		\item Unicellularr only
		\item Prokaryotes only
		\item Hetero-, Photoauto, Chemoauto-
	\end{itemize}
	\item \textbf{\emph{Archaea}}
	
	\begin{itemize}
		\item Unicellular only
		\item Prokaryotes only
		\item Hetero-, Chemoauto-
	\end{itemize}
	\item \textbf{\emph{Protists}}
	
	\begin{itemize}
		\item Both Multicellular and Unicellular
		\item Eukaryotes only
		\item Hetero-, Photoauto-
	\end{itemize}
	\end{itemize}
\end{itemize}

\subsection{Thur, Jan 30th: Organismal Origins}

\defn Proto-Cells: A Collection of lipids, A stepping stone to the origin of life.
\defn Biomolecules
\defn Evolutionary Theory: Describes change in living things over time. It explains the historical trajectory of individual groups of organisms. It also explains relatedness among diverse groups of organisms.
\defn Domain Bacteria
\defn Domain Archaea
\defn Domain Eukarya
\defn Common Ancestor
\defn Deep-sea Hydrothermal Vents: Geysers spewing hot water (760F!) full of nutrients and minerals, discovered in 1970's.
\defn Tube worms and trophosomes: A worms 'Gut'. It helped deal with the worms nutrients
\defn Cyanobacteria 
\defn Oxygenic Photosynthesis
\defn Mitochondria
\defn Aerobic Bacteria
\defn Cellular Respiration
\defn Chloroplasts
\defn Endosymbiotic Theory

\begin{itemize}

\item How do you go from a bunch of non-living chemicals floating around to the first living organism?
\item Proto-cells formed that contained 3 functional biomolecules surrounded by lipid membrane.
\begin{itemize}
\item Carbohydrates: Used for energy and structure.
\item Proteins: Used for structure and cell functions (e.g. enzymes, transport, defense, communication).
\item Nucleic Acids: Hold genetic instructions for making proteins
\end{itemize}
\item Proto-cells self replicate and slight changes occur each time.
\item Repeated 'Descent with Modification' took us from the first living organism to the huge diversity of organisms on earth.
\item The 3 domains of life, Bacteria, Archaea, and Eukarya are all united by a single Common Ancestor
\item Scientists hypothesize that the first living organisms were Chemoautotroph prokaryotes in deep-sea hydrothermal vents
\begin{itemize}
\item Early prokaryotes were likely chemoautotrophs because they lived in deep dark water and had no access to sunlight.
\end{itemize}
\item ~1.5 billion years after the first prokaryotes arose, the first photoautotroph prokaryotes evolved: Cyanobacteria!
\item Eukaryotic organisms arose after bacteria and archaea.
\item Eukaryotic cells are more complex than prokaryotic cells.
\begin{itemize}
\item DNA in Nucleus: centralized DNA for more control over protein production.
\item Mitochondria: Make ATP energy.
\item Chloroplasts: make food (photoautotrophs only)
\end{itemize}
\item Chloroplasts allow some eukaryotes to use sunlight to make food by performing oxygenic photosynthesis.
\item Mitochondria allows all eukaryotes to use food to make ATP energy through the process of cellular respiration.
\item How did eukaryotes get mitochondria? Through endosymbiosis of aerobic bacteria! The large prokaryote eats the small prokaryote.
\item How did eukaryotes get chloroplasts? Through the endosymbiosis of cyanobacteria!
\item Evidence supports endosymbiotic theory that mitochondria and chloroplasts were once free-living bacteria:
\begin{itemize}
\item They have their owm DNA and make their own proteins
\item They are genetically more closely related to free-living bacteria (that werent engulfed) than to their eukaryote hosts
\item They have double cell membranes (one from being single celled organisms and another formed when engulfed.
\item They reproduce independently of the eukaryotic cell.
\end{itemize}
\end{itemize}


\section{Febuary, 2020}

\section{March, 2020}

\section{April, 2020}

\section{May, 2020}



\end{document}